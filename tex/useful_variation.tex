\section{Bias towards Viable Variation} \label{sec:useful_variation}

In biology, the genetic information of an organism, its DNA, strongly influences the organism's phenotype via a developmental process. Hence, the genetic encoding is said to be indirect. This indirect genetic encoding provides a good example of a biological mechanism that promotes viable phenotypic variation; indirect encoding is hypothesized to exert an innate bias towards viable variation by promoting phenotypic regularity. Informally, regularity can be used to describe repetition of phenotypic form. Repetitive form might manifest as symmetry and/or recurring modular substructures. Formally, regularity refers to how much information is required to describe a structure \cite{Clune2011OnRegularity}. Indirect encodings tend to be biased towards phenotypic regularity because a large amount of phenotypic information is generated from a smaller amount of genetic information via a developmental process; these processes may favor regularity because each piece of genetic information determines many pieces of phenotypic information --- otherwise independent phenotypic characteristics are specified in a more coordinated fashion \cite{Clune2011OnRegularity}. 

This bias towards regularity in phenotypic form tends to translate to a bias towards viable variation. That is, in aggregate, regular phenotypes tend to outperform highly irregular phenotypes in most situations. The superior viability of regular phenotypes, of course, depends on the demands of the environment that the phenotype inhabits. Phenotypic regularity tends to be useful in more regular environments (that is, environments that exhibit regular characteristics), which are commonplace in the natural world \cite[pg 161]{Downing2015IntelligenceSystems}. Many domains of interest to EA researchers are also highly regular \cite{Clune2011OnRegularity}. Thus, digital and biological organisms often benefit from the bias towards regularity inherent to the indirect encoding of phenotypic characteristics in the genotype.

Figure \inputandref{direct_irregular_vs_indirect_regular}, taken from \cite{Cheney2013UnshacklingEncoding}, nicely illustrates the impact of indirect encodings on regularity. The figure compares two virtual soft robots, which are composed of an arrangement of small colored voxels with each color representing a unique tissue type. Each of the differentiated tissues has a distinct set of physical properties. This arrangement of voxels -- the organism's phenotype -- determines the fitness of the individual in its environment. In this case, fitness is determined by the individual's ability to walk on a simulated flat surface. The leftmost individual was generated via evolutionary search using a direct genetic encoding; the tissue type of each and every voxel was directly encoded in the individual's genome. In contrast, the rightmost individual was generated via evolutionary search using an indirect genetic encoding. Unlike its counterpart, the genetic information of this individual is encoded as a Compositional Pattern Producing Network (CPPN), a mathematical formula that translates three-dimensional spatial coordinates into tissue types. The phenotype of this individual, the spatial arrangement of tissue voxels, is determined by the \textit{output} of its genome, the CPPN. A striking visual contrast exists between the two soft robots. The indirect encoded individual is highly regular with large patches of uniform tissue type while the direct encoded individual is a jumble of different-colored voxels. As one might expect, the solution found by evolutionary search with indirect encoding, which likely shuffles along with a motion vaguely resembling an animal gait, moves faster than the solution found by evolutionary search with direct encoding, which likely moves at a slow crawl by vibration. In this particular virtual environment, a correlation between phenotypic regularity and fitness (i.e. the ability to move) biases the evolutionary search with the indirect encoding towards higher-fitness solutions. Indeed, evolutionary search in this soft robot locomotion domain with indirect genetic encoding was experimentally shown to vastly outperform search with direct genetic encoding \cite{Cheney2013UnshacklingEncoding}.