\section{Evolvability in a Broader Context} \label{sec:conclusion}

Evolvability is a topic of active discussion among elements of the evolutionary biology community  \cite{Pigliucci2008IsEvolvable}. The concept falls under the umbrella of a broader effort to expand the theoretical framework of evolution called the extended evolutionary synthesis \cite{Pigliucci2007DoSynthesis}. Compared to the EA community, however, the concept of evolvability has been slower to gain traction. As Kirschner and Gerhart, a pair of evolutionary biologists known for their theory of facilitated variation, comment,
\begin{displayquote}
``Many evolutionary biologists do not see a need to connect somatic adaptability to the generation of variation, and some see a need to keep them separate. For them, it is sufficient to say that random mutation is required and that the phenotypic variation arises haphazardly from it as random damage; the organism's current phenotype does not matter for the variation produced, and the output of variation is nearly random \cite[p 219]{Kirschner2005TheDilemma}.''
\end{displayquote}
Perhaps, in part, evolutionary biologists are less predisposed to interest in evolvability because they are not so directly stymied its absence. Success in attempts to emulate the evolutionary process to generate designs for sophisticated systems such as artificial neural networks or robotic bodies hinges on the ability of the evolutionary algorithm to generate viable, heritable variation. This development of this capability has been a major hurdle in EA research, especially in the field's early years. The intense and ubiquitous interest in evolvability among the EA community should therefore come as no surprise.

EAs have yielded interesting and useful results, but have not yet come close to replicating the intricacy or scale of biological systems \cite{Tonelli2011OnSystems}. In classical EAs, a directly or trivially indirectly encoded population evolves against a static fitness function. Fitness gains are typically realized for a period of several hundred generations before innovation stagnates and the population settles out at an equilibrium. This approach, predicated on a fundamentally accurate but extremely simplistic view of evolution, yields limited results. The stunted effectiveness of early EAs might be cast as a reflection of the limitations innate to the theory on which the algorithms are built. Mirroring activity among elements of the evolutionary biology community, there has been a fruitful thrust to build algorithms that incorporate a broader array of theoretical factors that may influence evolution, such as varying fitness functions and phenotypic plasticity \cite{Kashtan2007VaryingEvolution,Moczek2011TheInnovation,Downing2012HeterochronousBaldwinism}.

At present, it seems likely that evolvability stems from a large and diffuse web of cooperating mechanisms. The establishment --- or rejection --- of empirical evidence for causal links between factors such as plasticity or the developmental process and evolvability must be a key research goal in the field of evolutionary algorithm design. Such results will directly support efforts to refine the evolutionary algorithm and realize performance more closely akin to that of its biological counterpart. This line of inquiry raises and addresses questions of interest to the evolutionary biology community, especially in light of controversy surrounding the extended evolutionary synthesis. It will help determine which theoretical elaborations are necessary to account for evolution as observed in biology. It is hoped that further research in this vein --- both \textit{in silico} and \textit{in vivo} --- and, especially, continued exchange between EA and evolutionary biology researchers will yield both biological insight and more powerful digital engineering techniques.