% \section{Motivation and Definition}

The impressive matching of form to function in biological systems has long been admired by engineers, giving rise to the field of biomimicry, where design elements generated by the evolutionary process are employed in technological applications. Examples of biomimicry include legged locomotion in robotics that provides both efficiency and maneuverability \cite{Grimes2012THE}, nanotextures mimicking shark skin on boats that discourage barnacle growth while simultaneously decreasing water drag on the vessel \cite{Stenzel2011Drag-reducingShipping}, and tire treads inspired by the wet-adhesive properties of tree frog toe pads \cite{Persson2007WetTires}. Soon after the advent of modern computing, researchers began experimenting with biomimicry at a higher level of abstraction. Instead of mimicking the particular phenotypic forms generated through evolution, they harnessed the evolutionary process --- repeated cycles of selection on random variation --- to generate novel solutions to a wide array of problems. This approach has since blossomed into the field of evolutionary algorithm (EA) design \cite{Mitchell1996AnAlgorithms}. Language used to discuss EA reflects the biological metaphor on which the algorithm is predicated.

Evolutionary algorithms operate on candidate solutions to a problem, which in the biological metaphor are equivalent to \textit{individuals}. The aptitude of candidate solutions to solving a target problem is used to determine the candidate solution's \textit{fitness}, the amount of offspring it generates. Evolutionary algorithms traditionally begin with a \textit{population} of randomly-generated candidate solutions. Then, through a series of successive generations, the population is regenerated through \textit{recombination} of fit candidate solutions, so \textit{selection} is performed for candidate solutions that better satisfy the target problem. In biological evolution, a distinction is drawn between the \textit{phenotype} of an individual --- the physical characteristics which govern its interaction with the environment, its morphological, physiological, chemical, and molecular characteristics --- and the \textit{genotype} of an individual --- the heritable information that influences the phenotype displayed by the individual, i.e. the ordered sequence of base pairs in its DNA. This distinction can become blurred in the realm of evolutionary algorithms, where the phenotypic characteristics of an individual might be directly encoded in the genotype. We will return to this idea later on.

The desired outcome of the evolutionary algorithm is, as generations elapse, to observe candidate solutions that provide an increasingly satisfactory solution to the target problem that was used to determine their fitness. Once a predefined stopping criterion is met, usually after a specific number of generations or at a threshold fitness score, the evolutionary algorithm halts. Researchers and engineers have widely demonstrated the ability of EAs to attack labor-intensive optimization problems and to discover novel solutions beyond the reach of human ingenuity \cite{Poli2008AProgramming}. For example, evolutionary methods have been successfully applied to evolve communication antenna designs to satisfy the demanding specifications necessary for use in miniaturized spacecraft  \cite{Hornby2006AutomatedAlgorithms}. Figure \inputandref{evolved_antenna} depicts an evolved antenna design from that project. Although its form appears alien to traditional human approaches to design, it is nonetheless effective.

While biological phenotypic adaptation is indeed spectacular, another marvel of biology lurks just below the parade of phenotypes well-suited to their respective environments. It is hypothesized that biological organisms exhibit adaptation to the evolutionary process itself, not just to their environment over the course of their lifespans. That is, biological organisms are thought to possess traits that facilitate successful evolutionary search. The term evolvability was coined to describe such traits. A general consensus exists in the literature that evolvability stems from traits that facilitate the generation of heritable phenotypic variation that is viable.\footnote{This statement does not suggest that mutation is nonrandom, a controversial and widely discredited theory referred to biologists as adaptive mutation. Instead, it is predicated on the notion that the internal configuration of a biological system (i.e. the developmental process, modularity, degeneracy, etc.) constrains the outcomes of arbitrary perturbations to that system. It is hypothesized that biological organisms possess traits that influence the distribution of phenotypic effects of random mutation.} Breaking the concept down, evolvability stems from:
\begin{enumerate}
\item the amount of novel, heritable phenotypic variation among offspring,
\item the degree to which heritable phenotypic variation among offspring is viable,\footnote{This can be thought of in terms of the frequency at which lethal or otherwise severely harmful mutational outcomes are observed.}
\end{enumerate}
The dependence of evolution on these capacities is straightforward. Without any heritable variation, evolution would have no raw material to select from and would stagnate. Without any viable variation, evolution would select against all novelty and again stagnate. Hence, systematic evolutionary change depends the production of heritable, novel phenotypic variation, some of which must not be severely deleterious. We have established plausible traits that might facilitate evolution, but several important questions remain unanswered. How does evolvability manifest in biological organisms (i.e. what traits of biological organisms provide proximate explanations for the presence of viable heritable variation among offspring)? Why does evolvability manifest (i.e. what ultimate mechanistic forces endow biological organisms with traits that promote evolvability)? Addressing these two questions gives us a shot at tackling a third: how can evolvability be promoted in evolutionary algorithms? We will proceed to explore these questions, but let's begin by priming our intuition for evolvability by considering an artificial selection experiment performed on \textit{Drosophila melangoster}, common fruit flies.

These experiments, performed by Tuinstra et al. (1990) and Coyne (1987), revealed that bilaterally asymmetric phenotypic traits, such as different-sized eyes, could not be induced through artificial selection. In contrast, other artificial selection criteria, such as overall smaller eyes, yielded observable phenotypic changes over the course of a number of generations. The success of artificial selection for most traits on \textit{Drosophila} demonstrates the existence of a good amount of heritable phenotypic variation for those traits.  It is hypothesized that the negative result in artificial selection for bilaterally asymmetric phenotypic traits is due to a lack of bilateral symmetry-breaking information during the embryological development of \textit{Drosophila}. In other words, the very nature of the developmental process constrains the nature of phenotypic variation that can be observed in offspring, in this case curtailing the abundance of offspring that lack bilateral symmetry. As Tuinstra et al. (1990) phrase it, ``the developmental system does not seem to allow this type of variation.'' In the life of a fly, buzzing about in search of food and sex, bilateral symmetry is usually more fit than asymmetry. In this way, the distribution of phenotypic diversity in offspring is biased away from a particular type of deleterious variation, asymmetry. The results from these artificial selection experiments can be cast in terms of evolvability: the distribution of phenotypic outcomes of mutation is not entirely arbitrary. \textit{Drosophila melangoster}  more readily exhibits heritable phenotypic variation for certain traits --- overall eye size, for example --- than for other traits, such as bilateral asymmetry.

Armed with this cursory introduction to evolvability, let us continue our discussion of evolvability by focusing in on each of its two components separately: promotion of heritable phenotypic variation (Section \ref{sec:heritable_variation}) and bias against deleterious phenotypic variation (Section \ref{sec:useful_variation}). In Section \ref{sec:conclusion}, we will conclude by examining how evolvability fits into contemporary activity in the fields evolutionary biology and evolutionary algorithm design.