\section{Promotion of Heritable Variation} \label{sec:heritable_variation}

In discussing the generation of heritable variation, an important theoretical distinction can be made between individual evolvability and population evolvability. Individual evolvability refers to the potential of an individual to generate a diverse set of offspring \cite{Mengistu2016EvolvabilityIt}. In individuals with high individual evolvability, phenotypic form is less stable under genetic mutation. That is, mutations tend to yield more dramatic phenotypic change more frequently \cite{Wilder2015ReconcilingEvolvability}. In contrast, population evolvability refers to total amount of phenotypic diversity among potential offspring of a population as a whole \cite{Wilder2015ReconcilingEvolvability}. Although individual and population evolvability might be correlated to some extent, a direct relationship does not exist between the two. In the context of this paper, we will focus on discussion of individual evolvability.

Discussion of individual evolvability is predicated on the notion that biological organisms can possess qualities that facilitate heritable variation for a phenotypic trait. The regulatory action of hormonal signals such as somatotropin exemplify such a quality. This compound, also known as growth hormone, is well known for its widespread anabolic effects on tissues throughout the body. Mutations affecting the regulatory pathways that regulate somatotropin production and release, receptors and cell signaling components that mediate cellular response to somatotropin, and the protein itself all provide avenues for significant heritable variation in body size \cite{Devesa2016MultipleGrowth}.\footnote{Recent research implicates somatotropin in a number of processes unrelated to its classical association with metabolism and growth. Although the phenotypic consequences of mutations affecting somatotropin pathways are not exclusively limited to body size, somatotropin response nonetheless provides an avenue for heritable phenotypic variation in that regard.} The presence of such hormonal signaling pathways could be viewed as making a broad range of heritable phenotypic variation more readily realizable via mutation, increasing individual evolvabiltiy. Dog breeds, which exhibit a range of body weights nearly spanning an entire order of magnitude, evidence the accessibility of heritable variation for body size in animals. Among certain groups of dogs, much of this variation can be explained by just six genes, several of which are associated with pathways somatotropin participates in \cite{Rimbault2013DerivedBreeds.}.

Variation in environmental conditions over evolutionary time is thought to promote individual evolvability. This idea is motivated in part by the fact that biological organisms do not evolve in a static environment. Instead, their environment changes over evolutionary time. These changes might be due to abiotic factors, such as changes in the climate or the geological composition of an area. These changes might also be due to biotic interactions with other organisms, such as competition for resources. Thus, criteria that determine fitness --- which strategies for survival and reproduction are viable in a particular environment --- are temporally varying. 

Discussing a hypothetical example will help provide intuition for the concept of temporally varying fitness criteria and its consequences. Consider a hypothetical population of finches, which feed on a particular type of seed. Suppose that in order to successfully feed, finch beak widths must be matched with to the size of the seeds they feed on -- beaks can be neither too wide nor too narrow or the finch will be unable to feed effectively. Suppose that the sizes of the seeds that the hummingbird depends on were to be systematically manipulated over evolutionary time, proceeding through cycles of gradual increase and decrease. Under this  regimen, individuals that are predisposed to yielding variable offspring are advantaged over individuals that are not. Although much of that phenotypic variation is likely to be deleterious, it is likely at least some will prove adaptive. Thus, a subset of the offspring from individuals with high individual evolvability tend to outcompete the offspring from individuals with low individual evolvability that lack fresh adaptation to changing environmental conditions. As Wilder, et al. (2015) put it, ``if selection sets a moving target, individuals will be more likely to introduce variation in their offspring to adapt to an uncertain future.''  

Gradually shifting fitness criteria is thought to induce evolutionary pressure for individual evolvability, essentially providing a means of selecting for it. Evolutionary simulations have confirmed that gradual changes to the environment --- that is, a temporally varying fitness function --- can promote individual evolvability \cite{Kashtan2007VaryingEvolution, Wilder2015ReconcilingEvolvability}. By inducing a selective pressure for individuals with phenotypic variation among their offspring, some of which will track changing environmental conditions, temporally varying goals promote traits that facilitate the generation of heritable variation.